\documentclass{article}
\begin{document}

\title{5611 HW3: The Travelling Salesman Problem}
\author{Shane Harding \\ 10309405}

\maketitle

\section{Introduction}

This report details everything that I did in this assignment and why I did things the way I did.

\section{Task 1}

There are multiple C files to this task that are then all compiled together via the makefile.

The first thing I did was write a function that computes the distance between two points on a two dimensional plane (written in \verb!dist.c!). This does not give the proper Euclidean distance since I do not compute the square root (just $x^2 + y^2$ is calculated). The reason for this is that square root calculations are expensive and not necessary in this case. They aren't necessary because we are just comparing distances and if $d_1^2$ is bigger than $d_2^2$ then $d_1$ is bigger than $d_2$.

In my main file, the first thing I do is deal with the command line arguments and flags. So only one or zero flags are allowed. If there are no flags then the number of cities is simply set to 10. The \verb!-f! flag allows for a configuration of cities to be loaded from a specified file. The \verb!-n! flag generates a specified number of random cities on a grid.

Once all the coordinates of the cities are loaded/generated a distance matrix is set up. This puts the distance between cities $i$ and $j$ in the matrix entry $(i,j)$. The initial ordering of the cities set to be $0,1,...,n$. The permutation function is then called to generate all the possible routes. At this stage it is important to note that I am solving this problem for closed paths. So the first point and the last point are connected. This means that our starting city does not make any difference. Which then means that we don't have to move it in our permutations. And we also have $(n-1)!$ possible paths.

There is a swap function that simply swaps two elements in an array.

\section{Task 2}

\section{Task 3}


\section{Task 4}
Task 4 is this report. And I added the section before realising how stupid adding it was, and had a good laugh at myself.



\end{document}
